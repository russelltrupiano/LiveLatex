\documentclass{article}

\usepackage{graphicx}
\usepackage{algorithm}
\usepackage{algorithmic}
\usepackage{amssymb}
\usepackage{gensymb}
\usepackage{enumerate}
\usepackage{url}
\usepackage{amsmath}
\newcommand\ddfrac[2]{\frac{\displaystyle #1}{\displaystyle #2}}
\usepackage{changepage}
\usepackage[top=.5in, bottom=1in, left=1in, right=1in]{geometry}


\title {ASTRO 206 Homework 2}
\author{Russell Trupiano}
\begin{document}
\date{Due Tuesday, February 25, 2015}

\maketitlepoop

\section {Special Relativity (10 points)}
    \begin{adjustwidth}{.6cm}{}
        \begin{enumerate}[a)]
            \item
            Time dilation equation: $\Delta{t'} = \dfrac{\Delta{t}}{\sqrt{1-\dfrac{v^2}{c^2}}}$\\
            Solve for $v$:\\

            $\Delta{t'}\sqrt{1-\dfrac{v^2}{c^2}} = \Delta{t}$\\

            $(\Delta{t'})^2(1-\dfrac{v^2}{c^2}) = (\Delta{t})^2$\\

            $1-\dfrac{v^2}{c^2} = \dfrac{(\Delta{t})^2}{(\Delta{t'})^2}$\\

            $\dfrac{v^2}{c^2} = 1-\dfrac{(\Delta{t})^2}{(\Delta{t'})^2}$\\

            $v^2 = c^2\Bigg(1-\dfrac{(\Delta{t})^2}{(\Delta{t'})^2}\Bigg)$\\

            $v = c\sqrt{1-\dfrac{(\Delta{t})^2}{(\Delta{t'})^2}}$\\

            To live 8 times longer, the ratio $\Delta{t'}:\Delta{t}$ would be 8:1. Therefore, $\Delta{t'} = 8$ and $\Delta{t} = 1$.\\

            $v = 3e8\sqrt{1-\dfrac{1^2}{8^2}} = 3e8\sqrt{1-\dfrac{1}{64}}$\\

            $v = $ \textbf{2.976e8 m/s}

            \item
            Time dilation equation (using Lorentz factor): $\Delta{t'} = \gamma\Delta{t}$\\
            Solve for $\gamma$:\\

            $\gamma = \dfrac{\Delta{t'}}{\Delta{t}}$\\

            $\gamma = 2$\\

            This conclusion can be reached qualitatively as well considering that the person in motion is experiencing half the time as a person at rest, and that derivation was shown in $(a)$.\\

            Length contraction equation (using Lorentz factor): $L = \dfrac{L_0}{\gamma}$\\
            Solve for $L$:\\

            $L = \dfrac{1m}{2}$\\

            $L = $ \textbf{.5m}


        \end{enumerate}
    \end{adjustwidth}

\section {Gravitational Redshift (10 points)}
    \begin{adjustwidth}{.85cm}{}
        \textbf{Givens:}\\
        Gravitational redshift equation (given): $\dfrac{\lambda_2}{\lambda_1} = \Bigg[\dfrac{1-\dfrac{2GM}{{r_2}c^2}}{1-\dfrac{2GM}{{r_1}c^2}}\Bigg]^\frac{1}{2}$. \\
        $G = $ 6.67e-11 $m^3kg^{-1}s^{-2}$\\
        $c = $ 3e8 $ms^{-1}$\\
        $M_{sun} = M_{white\_dwarf} = $ 1.989e30 kg\\
        $r_{sun} = $ 6.95e8 m\\
        $r_{white\_dwarf} = (0.01)*r_{sun} = $ 6.95e6 m\\
        \\
        \textbf{Calculations:}\\
        Let $\lambda_1 = \lambda_{sun}$ and $\lambda_2 = \lambda_{white\_dwarf}$.\\
        $\dfrac{\lambda_2}{\lambda_1} = \Bigg[\dfrac{1-\dfrac{2GM}{{r_2}c^2}}{1-\dfrac{2GM}{{r_1}c^2}}\Bigg]^\frac{1}{2}$\\

        $ = \Bigg[\dfrac{1-\dfrac{2(6.67e{-11})(1.989e30)}{(6.95e6)(3e8)^2}}{1-\dfrac{2(6.67e{-11})(1.989e30)}{(6.95e8)(3e8)^2}}\Bigg]^\frac{1}{2}$\\

        $ = \Bigg[\dfrac{1-\dfrac{2948.14}{6.95e6}}{1-\dfrac{2948.14}{6.95e8}}\Bigg]^\frac{1}{2}$\\

        $ = \Bigg[\dfrac{1-4.24e{-4}}{1-4.24e{-6}}\Bigg]^\frac{1}{2}$\\

        $ = $ \textbf{.99979}\\

        \noindent
        This fact shows us that there is barely any effect on wavelengths from such a white dwarf due to gravitational redshift. Therefore, we would expect the wavelength from the same hydrogen line from the white dwarf to be \textbf{656.16 nm}: nearly identical to the original.
    \end{adjustwidth}
\section{Sirius Binary System (30 points)}
    \begin{adjustwidth}{.6cm}{}
        \begin{enumerate}[a)]
            \item
            $p_{Sirius} = $ .379 arc-seconds\\
            $d_{parsecs} = \dfrac{1}{p_{arc-seconds}}$\\
            $d_{parsecs} = \dfrac{1}{.379\:arcseconds} = $ \textbf{2.63 parsecs}\\
            \\
            $d_{lightyear} = 3.26*d_{parsecs}$\\
            $d_{lightyear} = 3.26*2.63\:parsecs = $ \textbf{8.60 light years}
            \clearpage
            \item
            Using the scale in the figure, I approximated the semi major axis of Sirius B about Sirius A to be about 7 arcseconds. From there we can use tigonometry to solve for the semimajor axis in astronomical units.\\
            Givens:\\
            $p = 7\:arcseconds = \Bigg(\dfrac{7}{3600}\Bigg)\degree = .00194\degree$\\
            $d = 2.63\:pc$ (From $(a)$)\\
            $tan(p) = \dfrac{a}{d}$\\

            Solve for $a$:\\
            $a = d*tan(p) = 2.63*tan(.00194) = 8.91e{-5}\:pc$\\
            $1\:pc = 206264\:AU$\\
            $a = 8.91e{-5}*206264 =$ \textbf{18.37 AU}
            \item
            $\lambda_{Sirius\:A} = $ 292 nm\\
            Wein's Law: $\lambda_{max} \approx \dfrac{2.9e6}{T(kelvin)}$nm\\
            $T \approx \dfrac{2.9e6}{292} \approx $ \textbf{9931 K}\\
            Sirius A is much hotter than the sun which has a temperature of 5778 K.\\
            Sirius A is a class-A star.
            \item
            From Figure 2, a line drawn straight up from about the 10,000K marker on the x-axis itersects the white dwarf luminosity curve at about $20*L_{\bigodot}$. Therefore we would estimate the luminosity of Sirius A to be approximately $2.11*M_{\bigodot} = $ \textbf{4.20e30 kg}.
            \item
            Sirius B's period of its orbit around Sirius A is approximately \textbf{50 years}.
            \item
            Kepler's 3rd Law Generalized: $P^2 = \dfrac{4{\pi}^2}{G(M_A + M_B)}a^3$\\
            Solving for $M_B$:\\

            $\dfrac{GP^2}{4a^3{\pi}^2} = \dfrac{1}{(M_A + M_B)}$\\

            $\dfrac{4a^3{\pi}^2}{GP^2} = M_A + M_B$\\

            $\dfrac{4a^3{\pi}^2}{GP^2}-M_A = M_B$\\

            $M_B = \dfrac{4(18.37AU*1.5e11\frac{m}{AU})^3{\pi}^2}{(6.67e{-11})(50yr*3.16e7\frac{s}{year})^2}-4.2e30kg$\\

            $M_B = $ \textbf{7.60e29 kg}\\

            This answer is slightly off from the actual mass of Sirius B, but it is likely to do with the approximation method used in $(d)$; however, it does seem to be the right order of magnitude, and also Sirius A was approximated slightly higher than it should have been.
            \clearpage
            \item
            Relationship between magnitude and luminosity: $m_A - m_B = -2.5log_{10}\Bigg(\dfrac{L_A}{L_B}\Bigg)$\\
            $\Delta{m} = m_A - m_B = 7.5$\\

            $L_A = 25.4L_{\bigodot}$\\

            Solve for $L_B$:\\

            $\Delta{m} = -2.5log_{10}\Bigg(\dfrac{L_A}{L_B}\Bigg)$\\

            $\dfrac{L_A}{L_B} = 10^{\Delta{m}/2.5}$\\

            $L_B = \dfrac{L_A}{10^{\Delta{m}/2.5}} = \dfrac{25.4}{10^{3}}$\\

            $L_B = $ \textbf{.0254} $L_{\bigodot}$

            \item
            Luminosity equation: $L = 4{\pi}r^2{\sigma}T^4$\\

            $\sigma = 5.67*10^{-8}\frac{W}{m^2K^4}$\\

            $L = .0254\:L_{\bigodot} = 9.77e24\:W$ (From $(h)$)\\

            Solve for $r$:\\

            $r = \sqrt{\dfrac{L}{4{\pi}{\sigma}T^4}}$\\

            $r = \sqrt{\dfrac{9.77e24}{4{\pi}{(5.67*10^{-8})({25193}^4)}}}$\\

            $r = 5.84e6$ m$ = $ \textbf{5840 km}\\
            \item
            With this information we can calculate the density of Sirius B.\\

            $m = 7.60e32\:g$ and $V = \frac{4}{3}\pi r^3 = 8.34e26\:{cm}^3$ (From $(f)$ and $(h)$)\\

            $\rho = \dfrac{m}{V}$\\

            $\rho = \dfrac{7.60e32}{8.34e26} = 9.11e5\frac{g}{cm^3}$\\

            Knowing that the average density of a neutron star is $2e14\frac{g}{cm^3}$ and a black hole even more dense, we know that it would be unreasonable to classify Sirius B as such. Additionally, from its small radius we can then conclude that Sirius B is in fact a white dwarf.
        \end{enumerate}
    \end{adjustwidth}

\end{document}